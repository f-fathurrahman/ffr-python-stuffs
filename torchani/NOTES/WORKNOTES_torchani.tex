\documentclass[a4paper,12pt]{article} % screen setting

\usepackage[a4paper]{geometry}
\geometry{verbose,tmargin=1.5cm,bmargin=1.5cm,lmargin=1.5cm,rmargin=1.5cm}

\setlength{\parskip}{\smallskipamount}
\setlength{\parindent}{0pt}

%\usepackage{cmbright}
%\renewcommand{\familydefault}{\sfdefault}

%\usepackage{fontspec}
\usepackage[libertine]{newtxmath}
\usepackage[no-math]{fontspec}
\setmainfont{Linux Libertine O}
%\setmonofont{DejaVu Sans Mono}

\usepackage{hyperref}
\usepackage{url}
\usepackage{xcolor}

\usepackage{amsmath}
\usepackage{amssymb}

\usepackage{graphicx}
\usepackage{float}

\usepackage{minted}

\newminted{julia}{breaklines,fontsize=\footnotesize}
\newminted{python}{breaklines,fontsize=\footnotesize}

\newminted{bash}{breaklines,fontsize=\footnotesize}
\newminted{text}{breaklines,fontsize=\footnotesize}

\newcommand{\txtinline}[1]{\mintinline[breaklines,fontsize=\footnotesize]{text}{#1}}
\newcommand{\jlinline}[1]{\mintinline[breaklines,fontsize=\footnotesize]{julia}{#1}}
\newcommand{\pyinline}[1]{\mintinline[breaklines,fontsize=\footnotesize]{python}{#1}}

\newmintedfile[juliafile]{julia}{breaklines,fontsize=\footnotesize}
\newmintedfile[pythonfile]{python}{breaklines,fontsize=\footnotesize}

\definecolor{mintedbg}{rgb}{0.90,0.90,0.90}
\usepackage{mdframed}
\BeforeBeginEnvironment{minted}{
    \begin{mdframed}[backgroundcolor=mintedbg,%
        topline=false,bottomline=false,%
        leftline=false,rightline=false]
}
\AfterEndEnvironment{minted}{\end{mdframed}}


\usepackage{setspace}

\onehalfspacing

\usepackage{appendix}


\newcommand{\highlighteq}[1]{\colorbox{blue!25}{$\displaystyle#1$}}
\newcommand{\highlight}[1]{\colorbox{red!25}{#1}}



\begin{document}


\title{TorchANI Worknotes}
\author{Fadjar Fathurrahman}
\date{}
\maketitle

\section{Installation}

Prerequisites: PyTorch

Recommendation: install to user directory.

The usual steps are applicable:
\begin{itemize}
\item \txtinline{python setup.py build}
\item \txtinline{python setup.py install --user}
\end{itemize}

TorchANI will be installed at (for example, the version number
may vary):
\begin{textcode}
$HOME/.local/lib/python3.8/site-packages/torchani-2.1.2-py3.8.egg/
\end{textcode}

To check the installation, go to directory examples and run the
\txtinline{energy_force.py} example.
TorchANI will download the necessary files (paramaters for the neural network)
so the script might run slowly in the first call.
The parameters will be put under the \txtinline{resources} subdirectory of TorchANI.

There is warning when using Torch v-1.6:
\begin{textcode}
/home/efefer/.local/lib/python3.8/site-packages/torchani-2.1.2.dev12+g42442af-py3.8.egg/torchani/aev.py:195: UserWarning: This overload of nonzero is deprecated:
	nonzero()
Consider using one of the following signatures instead:
	nonzero(*, bool as_tuple) (Triggered internally at  /opt/conda/conda-bld/pytorch_1595629395347/work/torch/csrc/utils/python_arg_parser.cpp:766.)
in_cutoff = (distances <= cutoff).nonzero()
\end{textcode}

The fix seems to be:
\begin{pythoncode}
in_cutoff = torch.nonzero(distances <= cutoff, as_tuple=False)
\end{pythoncode}

\section{CPU vs GPU comparison}

Script \txtinline{test_cnt_5x5.py}. Timing using time command in Linux.
\begin{pythoncode}
from ase.build import nanotube
from ase.md.langevin import Langevin
from ase.optimize import BFGS
from ase import units

import torch
import my_torchani

# Now let's set up a crystal
atoms = nanotube(5, 5, length=10, bond=1.420, symbol='C')
print(len(atoms), "atoms in the cell")
atoms.set_pbc([True,True,True])
Lz = atoms.cell.array[2,2]
Lx = 20.0
Ly = 20.0
atoms.set_cell([Lx, Ly, Lz])
atoms.center()
atoms.write("STRUCT_cnt_5x5.xsf")
    
#device = torch.device("cuda" if torch.cuda.is_available() else "cpu")
device = "cpu" # set manually to cpu
print("device = ", device)
    
# Now let's create a calculator from builtin models:
calculator = my_torchani.models.ANI1ccx().to(device).ase()
    
# Now let's set the calculator for ``atoms``:
atoms.set_calculator(calculator)
    
# Now let's minimize the structure:
print("Begin minimizing...")
opt = BFGS(atoms)
opt.run(fmax=0.001)
print()
    
    
# Now create a callback function that print interesting physical quantities:
def printenergy(a=atoms):
    # Function to print the potential, kinetic and total energy.
    epot = a.get_potential_energy() / len(a)
    ekin = a.get_kinetic_energy() / len(a)
    print('Energy per atom: Epot = %.3feV  Ekin = %.3feV (T=%3.0fK)  '
          'Etot = %.3feV' % (epot, ekin, ekin / (1.5 * units.kB), epot + ekin))

# We want to run MD with constant energy using the Langevin algorithm
# with a time step of 1 fs, the temperature 300K and the friction
# coefficient to 0.02 atomic units.
dyn = Langevin(atoms, 1 * units.fs, 300 * units.kB, 0.2)
dyn.attach(printenergy, interval=1)

# Now run the dynamics:
print("Beginning dynamics...")
printenergy()
dyn.run(500)    
\end{pythoncode}

Timing result:
\begin{textcode}
# CPU
real	1m35.680s
user	1m26.560s
sys	0m9.126s
    
# GPU
real	0m18.612s
user	0m15.110s
sys	0m3.454s
\end{textcode}


\bibliographystyle{unsrt}
\bibliography{BIBLIO}

\end{document}
