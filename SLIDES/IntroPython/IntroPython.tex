\documentclass[bahasa,10pt]{beamer}

\setlength{\parskip}{\smallskipamount}
\setlength{\parindent}{0pt}

\setbeamersize{text margin left=5pt, text margin right=5pt}

\usepackage{amsmath}
\usepackage{amssymb}
\usepackage{braket}

\usepackage{minted}
\newminted{python}{breaklines,fontsize=\footnotesize,texcomments=true}
\newminted{bash}{breaklines,fontsize=\footnotesize,texcomments=true}
\newminted{text}{breaklines,fontsize=\footnotesize,texcomments=true}

\definecolor{mintedbg}{rgb}{0.95,0.95,0.95}
\usepackage{mdframed}

\BeforeBeginEnvironment{minted}{\begin{mdframed}[backgroundcolor=mintedbg]}
\AfterEndEnvironment{minted}{\end{mdframed}}

\setcounter{secnumdepth}{3}
\setcounter{tocdepth}{3}

\makeatletter
%%%%%%%%%%%%%%%%%%%%%%%%%%%%%% Textclass specific LaTeX commands.
 % this default might be overridden by plain title style
 \newcommand\makebeamertitle{\frame{\maketitle}}%
 % (ERT) argument for the TOC
 \AtBeginDocument{%
   \let\origtableofcontents=\tableofcontents
   \def\tableofcontents{\@ifnextchar[{\origtableofcontents}{\gobbletableofcontents}}
   \def\gobbletableofcontents#1{\origtableofcontents}
 }

\makeatother

\usepackage{babel}

\begin{document}


\title{Dasar Pemrograman Python}
\author{Fadjar Fathurrahman}
\institute{
Program Studi Teknik Fisika \\
Divisi Komputasi Pusat Penelitian Nanosains dan Nanoteknologi \\
Institut Teknologi Bandung
}
\date{7 April 2018}

\frame{\titlepage}

\begin{frame}
\frametitle{Pemrograman Python}

Python banyak digunakan di ...

Kelebihan dan kekurangan Python ...

\end{frame}


\begin{frame}[fragile]
\frametitle{Shell interaktif dan script}

\begin{itemize}

  \item Kode program Python ditulis dalam file teks dengan ekstensi \texttt{.py}.
  Untuk menjalankannya kita dapat menggunakan perintah berikut pada terminal:
\begin{bashcode}
python nama_script.py
\end{bashcode}

  \item Shell interaktif: Dapat dijalankan dengan mengetikkan \texttt{python} pada
  terminal. Pada terminal akan muncul tampilan sebagai berikut.
\begin{textcode}
Python 3.6.1 |Anaconda custom (64-bit)| (default, May 11 2017, 13:09:58) 
[GCC 4.4.7 20120313 (Red Hat 4.4.7-1)] on linux
Type "help", "copyright", "credits" or "license" for more information.
>>>
\end{textcode}
  Kode Python kemudian dapat diketikkan setelah prompt \texttt{>>>}.

\end{itemize}

\end{frame}


\begin{frame}[fragile]
\frametitle{Contoh penggunaan shell interaktif}

\begin{textcode}
Python 3.6.1 |Anaconda custom (64-bit)| (default, May 11 2017, 13:09:58) 
[GCC 4.4.7 20120313 (Red Hat 4.4.7-1)] on linux
Type "help", "copyright", "credits" or "license" for more information.
>>> 2 + 4
6
>>> b = 3*2 - 1.1
>>> c = b/2;
>>> print("b = ", b, " c = ", c)
b =  4.9  c =  2.45
>>> exit()
\end{textcode}

\end{frame}


\begin{frame}[fragile]
\frametitle{\texttt{Ipython}: Enhanced Interactive Python}

\begin{textcode}
Python 3.6.1 |Anaconda custom (64-bit)| (default, May 11 2017, 13:09:58) 
Type "copyright", "credits" or "license" for more information.

IPython 5.4.1 -- An enhanced Interactive Python.
?         -> Introduction and overview of IPython's features.
%quickref -> Quick reference.
help      -> Python's own help system.
object?   -> Details about 'object', use 'object??' for extra details.

In [1]: 1.1 + 2.3
Out[1]: 3.4

In [2]: %quickref

In [3]: pwd
Out[3]: '/home/efefer'
\end{textcode}

\end{frame}

\begin{frame}
\frametitle{Ipython Notebook}

Demo

\end{frame}


\begin{frame}[fragile]
\frametitle{Numpy: notasi array mirip dengan MATLAB}

Demo

\end{frame}


\begin{frame}[fragile]
\frametitle{Matplotlib: plotting}
  
Demo
  
\end{frame}

\begin{frame}[fragile]
\frametitle{SciPy:}

Demo

\end{frame}


\begin{frame}[fragile]
\frametitle{sympy}
  
Demo
  
\end{frame}


\begin{frame}[fragile]
\frametitle{Lainnya}
    
GUI: Tk, PyGTK, PyQt

Web framework

\end{frame}



\begin{frame}[fragile]
\frametitle{Contoh script Python}

\begin{pythoncode}
# Contoh komentar (perbaris)

from __future__ import print_function # komentar jg bisa di sini

"""
Fungsi sederhana: menerima input dua bilangan dan mengembalikan
hasil penjumlahan dua bilangan tersebut
"""
def myfunc(a,b):
    c = a + b
    return 2*c

print('%d + %d = %d' % ( a, b, myfunc(a,b) ))
\end{pythoncode}


\end{frame}

\end{document}

