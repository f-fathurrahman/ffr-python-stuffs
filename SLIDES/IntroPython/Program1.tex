\section{Pemrograman Python}

\begin{frame}[fragile]
\frametitle{Konvensi}

Potongan kode Python pada bagian ini dapat sebagian dimaksudkan untuk
ditulis dalam file (script) sedangkan kode yang lain dapat langsung
diketik di Python atau IPython console.

Contoh kode Python dalam console:
\begin{pyconcode}
>>> 4 + 5
9
>>> 4.1 + 4.2
8.3
>>> 4.1*2
8.2
>>> 4/2
2.0
>>> 4/3
1.3333333333333333
\end{pyconcode}

\end{frame}


\begin{frame}[fragile]
\frametitle{Kode sumber Python}

Kode sumber Python ditulis dalam file teks dengan ekstensi \texttt{.py}.

Dari terminal dapat digunakan perintah
\begin{bashcode}
python namafile.py
\end{bashcode}

Sebagai contoh, buatlah file dengan nama \texttt{halo.py} atau nama lain
yang Anda suka dengan isi sebagai berikut:
\begin{pythoncode}
nama = "efefer"
print("Halo, nama saya adalah ", nama)
\end{pythoncode}

\end{frame}


\begin{frame}[fragile]
\frametitle{Komentar}

Komentar dapat ditulis dengan beberapa cara:
\begin{pythoncode}
# Komentar dalam satu baris
# Komentar lagi

"""
Komentar yang dapat digunakan dalam lebih dari satu baris.
Juga dapat diinterpretasikan sebagai string.
"""
\end{pythoncode}

\end{frame}


\begin{frame}[fragile]
\frametitle{Variabel dan tipe data}

Variabel dapat didefinisikan dengan menggunakan operator penugasan
(assigment) \texttt{=}

Tipe variabel dalam Python tidak perlu dituliskan. Python akan menebak
tipe data tersebut secara otomatis.

\begin{pyconcode}
>>> a = 2.4;  # gunakan tanda a; untuk mencegah penulisan output
>>> b = 3;
>>> type(a) # type: mengetahui tipe dari suatu variabel
<class 'float'>
>>> type(b)
<class 'int'>
\end{pyconcode}

\begin{pyconcode}
>>> 5 // 3
1
>>> 5.0 // 3.0
1.0
>>> 10 % 3
1
>>> 2**4
16
>>> 2.0**4
16.0
\end{pyconcode}

\end{frame}



